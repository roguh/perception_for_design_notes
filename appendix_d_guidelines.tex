\newcommand{\e}[1]{\textbf{#1}}

\section{Visualization Guidelines}

%\begin{multicols}{2}

\subsection{Chapter 1}

\e{Foundation for a science of data visualization.}

\begin{compactenum}

    \item Design graphic representations of data by taking into account human
        sensory capabilities in such a way that \e{important} data
        elements/patterns can be \e{perceived}

    \item Important data should be visually distinct (which items are more
        findable?)

    \item Visual distinction should match the data's magnitude

    \item Graphical symbols should be \e{standardized} within and across
        applications

    \item Choose the tool that allows for the most work per time,
        \e{save time}

    \item Novel solutions are good if they outweigh the \e{cost of learning}

    \item Unless novelty outweighs inconsistency, pick consistent tools

    \item Effort spend on development should equal expected ``profit''

\end{compactenum}

\subsection{Chapter 2}
\e{The environment, optics, resolution and the display}

\begin{compactenum}

    \item \e{Lambertian} shading to reveal shapes of smooth surfaces

    \item Use \e{specular} shading to reveal fine surface details. Object/light
        motion should be enabled.

    \item \e{Cast shadows} for large-scale spatial relationships. Do not use
        unless the visual cost is made up for in benefits.

    \item Ambient \e{occlusion} for 2D shape perception, where no shading info
        is provided.

    \item Agumented-reality: info linked to an external object should be at the
        \e{same focal distance}

    \item Augmented reality: if info is unrelated to externals, the aug-info
        should seem \e{closer than everything}

    \item Head-mounted display: text should be no more than \e{18 degrees}

    \item \e{40 degree} viewing angle for data analysis

    \item wrap-around displays to give a sense of ``presence''

    \item Avoid high-contrast grating patterns, especially if flickering
        between \e{5Hz and 50Hz}.

    \item \e{Antialias} whenever possible, especially if there are fine details

\end{compactenum}

\subsection{Chapter 3}

\e{Lightness, brightness, contrast, constancy}

\begin{compactenum}

    \item \e{Avoid gray} scale as a method for representing more than 2 -- 4 numerical
    values

    \item Consider using \e{cornsweet contours} instead of simple lines to define
    convoluted (crazy) bounded regions

    \item Consider using \e{luminance contrast} as a highlighting method. Either
    reduce contrast of unimportant items or increase local background's
    contrast around important items (haloing).

    \item Use  minimum \e{3:1 luminance contrast} ratio when details such as
    texture variation, small-scale patterns or text exist. 10:1 is optimal
    for text.

    \item If \e{subtle grey-level} variations are important, create
        low-luminance contrast between object and its background.

    \item A light neutral colored wall \e{behind the screen} should reflect
        an amount of light comparable to that emitted by the screen.
        The wall facing the screen should be dark and of low reflectance.
        Avoid shining lights on the monitor.

    \item Minimal light should fall on the projector screen. \e{Shields},
        low-reflectance (mid to dark gray) walls are desirable.

\end{compactenum}

\subsection{Chapter 4}

\e{Color}

\begin{compactenum}

    \item Use more \e{saturated colors} for small symbols/lines/areas. Less
        saturated for larger areas

    \item Small symbols with a color different from the background's should
        have a \e{luminance contrast} (See 3.4)

    \item Adequate luminance contrast for the perception of \e{stereoscopic
        depth}

    \item Adequate luminance contrast for the perception of \e{motion}

    \item If shading to define a \e{curved surface}, use luminance instead
        of chroma. (See 2.1)

    \item If large areas have equiluminous coloring, use \e{thin borders} with
        large luminance differences

    \item Saturation should be used to encode \e{no more than 3} values

    \item Red-green, yellow-blue, white-black are independent.

    \item In an interface for design, allow the user to \e{change background} to
        see effects on coloring

    \item Easy-to-remember color codes: provide a \e{color palette}

    \item Pure red, green, yellow and blue to color code \e{small symbols}.

    \item Small color-coded symbols: ensure \e{luminance and chromatic} contrast
        with background

    \item Colored symbols isoluminant against parts of background? Add a
        \e{border} with a highly contrasting luminance (relative to symbol).

    \item To ensure color-blind accesibility, use \e{yellow-blue} variance.

    \item \e{No more than 10 colors} if precise identification is desired,
        especially if varying backgrounds.

    \item \e{Low saturation colors for large} areas. Light colors are best, more
        of these available.

    \item Large backgrounds overlaid with small colored symbols?
        High-value/\e{pastel colors for background, high saturation} for
        symbols

    \item If text is highlighted, ensure \e{luminance contrast} with background.

    \item Cultures have a \e{similar high-importance color} sequence, varying
        low-importance. Consider this in maps.

    \item If extremes or other patterns must be seen, use a
        color sequence with \e{monotonically increasing luminance}. Carefully
        space colors if discriminate steps are wanted

\end{compactenum}

\subsection{Chapter 5}

\e{Visual salience and finding information}

\begin{compactenum}

    \item Minimize the cost of visual searches by making displays as compact
        as possible. \e{Saccades of 5 degrees} or less.

    \item Visually distinct aspects should be represented by \e{distinct channels}.

    \item Symbols should be \e{distinct from background and other symbols}, if
        search is needed.

    \item Make symbols as \e{distinct from each other} as possible; orientation and
        spatial frequency (size difference).

    \item Same for symbols and background patterns.

    \item Strong preattentive cues \e{before weak ones} if search is critical.

    \item Maximum popout: a symbol should be the \e{only object in a channel}

    \item \e{Asymmetric preattentive} cues for highlighting

    \item Highlights should occupy the \e{least used feature dimension}.

    \item Use \e{subtle blinking or motion} if color and shape channels are full.

    \item Use \e{redundant coding} where possible

    \item If symbols are to be preattentively distinct, avoid using
        \e{conjunctions} of graphical properties.

    \item If two distinct attributes must be highlighted, consider using
        motion and color/shape or another \e{valid conjunctive pair}.

    \item If variables need to be seen holistically, map the variables to
        \e{integral} glyph properties

    \item If variables need to be seen analytically, map the variables to
        \e{separable} glyph properties

    \item \e{Quantity} shown by size, lightness on a dark background, darkness
        on light background, color saturation, vertical position

    \item Ideally, use \e{length or height} to represent quantity; maybe
        area. Never use 3D volume.

    \item Combine heterogeneous display channels with meaningful \e{mappings
        between} data and graphical features

    \item User interrupts must be made stronger if \e{cognitive load} is high.

\end{compactenum}

\subsection{Chapter 6}

\e{Static and moving patterns}

\begin{compactenum}

    \item Symbols with related information should maintain \e{proximity}

    \item Grid layout = low visual channel properties for \e{rows/columns}.

    \item Use lines or ribbons of color to show \e{relationships}.

    \item Use symmetry to make pattern comparisons easier, be sure patterns are
        $>2$ degrees horizontally, and $>1$ degree vertically. Symmetry should
        be along horizontal/vertical axes.

    \item Related information inside a \e{closed contour}.

    \item \e{Overlapping regions} = combine line contour, color, texture,
        Cornsweet contours.

    \item Combination of closure, common region and layout to ensure
        data entities represented by patterns will be \e{figure and not ground}.

    \item Use contours \e{tangential} to streamlines to reveal orientation.

    \item Flow direction in vector field = use \e{streamlets}.

    \item \e{Vector field} = more distinct elements indicate greater strength or
        speed. Wider, longer, more contrasting, faster moving.

    \item \e{Texture for continuous} map variables. Most effect when data varies
        smoothly.

    \item To make nominal coding textures \e{distinct}, make them differ as much as
        possible in orientation and element size. Also vary randomness in
        spacing of elements.

    \item \e{Simple texture} parameters such as element size/density only when
        fewer than five ordinal steps must be distinguished

    \item \e{Bivariate scalar} field? Map one to color and one to texture
        variations.

    \item Design textures so quantitative values can be judged? Visually order
        so that \e{small} can be differentiated from \e{large}.

    \item When using overlapping textures to separate overlapping regions,
        avoid patterns that can cause \e{aliasing} when combined

    \item Textures in combination with background colors in overlapping
        regions? Use \e{lacy textures}.

    \item Lacy textures in combination with colors for \e{overlapping} regions.
        Ensure luminance contrast between texture in foreground and
        color-coded data in background.

    \item Discrete data with more than four dimensions? Use color-enhanced
        generalized \e{draftsman's plots with brushing}.

    \item \e{Standardize mappings} of data to patterns within and across
        applications.

    \item Search for infrequent targets? Insert \e{retraining sessions} and give
        feedback on failure/success.

    \item Glyphs = small enclosed \e{shapes} for entity. \e{Color, shape and size} for
        attributes.

    \item Connecting \e{lines, enclosure, grouping and attachment} to show
        relationships between entities. Shape/color/thickness for types.

    \item Alternative to arrows in directed graphs = \e{tapered lines} with
        broadest end at source node.

    \item Use \e{closed contours}, areas of texture or color to denote regions.
        Color/texture/boundary style to show type.

    \item Use \e{lines} to represent paths. Line color for type of linear feature.

    \item Use \e{small, closed shapes} to represent point entities (cities, etc).
        Use color, texture or boundary style to show attributes.

    \item Use treemap if only \e{leaf nodes and a quantity} they are associated
        with matter

    \item Node-link representation when \e{hierarchy and non-leaf nodes} matter and
        quantitative attributes are less important.

    \item Animation? Motion between \e{0.5 to 4 degrees} of visual angle.

\end{compactenum}




\subsection{Chapter 7}

\e{Visual objects and data objects}

\begin{compactenum}

\item 3D size judgments? Use the \e{best possible set} of depth cues.

\item Minimize perceived distortions from off-axis 3D spaces, avoid \e{wide
    viewing angles}, horizontal viewing angle below 30 degrees.

\item 3D height field data? Use \e{draped grids} to enhance shape information.
    Best if data varies smoothly.

\item Use \e{cast shadows} to tie objects to a surface that defines depth. Surface
    should have strong depth cues (grid texture, etc) and be simple, objects
    should be close to it

\item 3D depth relationships? Use structure-from-motion by rotating the scene
    around point of interest.

\item Stereoscopic? Avoid placing objects \e{too close} to the screen where their
    edges are clipped.

\item 3D stereoscopic? Use highest possible horizontal resolution. Excellent
    \e{spatial and temporal antialiasing}

\item Adjust \e{virtual eye separation} to optimize perceived depth while
    minimizing diplopia.

\item 3D with strong and gridded ground plane? Use \e{drop lines} to add depth
    information for small numbers of discrete isolated objects.

\item 3D? Use \e{halos} to enhance occlusion if this is an important depth cue and
    overlapping objects have same color/luminance

\item 3D? Use the depth cues for the most \e{critical tasks}.

\item \e{3D node-link structures}? Use motion parallax, stereoscopic, halos

\item Textures to reveal surface shapes especially if in stereo. Only good for
    smooth surfaces, and texture is otherwise unused. Linear components, if
    one texture is on top of another, make it lacy and see-through

\item Rotating \e{surface and stereoscopic} viewing to enhance 3D shape.

\item \e{Bivariate scalar} field maps. Use shaded height field and color for
    the two. Shaded variable should be relatively smooth.

\item \e{Depth in 3D scatterplot}, rotate or oscillate a point cloud around a
    vertical axis. Use stereoscopic if possible.

\item 3D \e{cloud of points}, must judge morphology of boundary? Use cloud surface
    to shade points.

\item 3D trajectories = \e{shaded tube or box} with periodic bands for orientation
    cues. Motion parallax and stereoscopic if possible.

\item Stereoscopic viewing when \e{hand movements} are critical. Use proxy for
    hand if possible.

\item 3D environments with 1-to-1 mapping between hand and objects, ensure
    \e{relative positions} are correct. Minimize rotational mismatch
    ($>30$ degrees).

\item Vertical polarity in 3D space = clear reference \e{ground} plane and
    gravity affected \e{familiar objects} on it

\item Vivid sense of \e{presence} in 3D = large FOV, smooth motion, lots of detail

\end{compactenum}




\subsection{Chapter 8}

\e{Space perception and the display of data in space}

\begin{compactenum}

\item Optimal identification = make important patterns, complex objects have
    a 4 to 6 degree angle. Only a gradual penalty if not so.

\item Use an object display if standardized sets of data must be analyzed
    repeatedly and data can be mapped to objects

\item Design object displays such that numbers are tied to recognizable objects
    representing components

\item Design layouts using connecting elements that clearly indicate
    physical connections between components

\item Design glyphs to have emergent properties revealing effect of important
    interactions between variables

\item Display glyphs should be more salient when critical values are reached

\item Represent components with geons (smooth and shaded spheres, cylinders,
    cones, boxes)

\item Use color/texture of geons to represent secondary objects

\item Fewer than 30 components, entities/relationships must be shown = use
    geon diagram

\item Represent relationship by joints, tubes. Small geon attached to bigger
    one to show component relationship.

\item Geon (easy to perceive, 3D) shapes to represent primary attribute of entities

\item 3D diagrams = lay as much as possible in 2D plane orthogonal to line
    of sight. Make sure all connections are visible.

\item 3D diagram = use transparent bag to express part-of relationhips

\item Entities and links = size/thickness to represent strength.

\item Efficient and compact expression of human emotion, use emoticons

\item Visual image that represents a class of things = use canonical example
    in its ``normal'' orientation but only if it exists.

\item Use icons for pedagogical purposes in infographics. Use only when no
    canonical or cultural image is available

\item Large number of data points = symbols instead of words/icons

\item Use words directly where the numbers of symbolic objects in each
    category is few and space is available

\item use Gestalt principles of proximity, connectedness and common region to
    link label and figure

\end{compactenum}




\subsection{Chapter 9}

\e{Images, words and gestures}

\begin{compactenum}

\item Use methods based on natural language (not visual patterns) to express
    detailed program logic

\item Graphical elements should be used to represent structure, like links.

\item Use methods based on natural language (not visual patterns) to represent
    abstract concepts

\item Complex information: separate components by their most efficient
    representations: words (spoken/written) or images (static/moving). Use the
    most cognitively efficient method for linking.

\item Exploratory text should be as close as possible and graphically linked

\item Presentations: speech should accompany graphics, not text

\item Use deixis (arrow) to link speech and graphics

\item Spoken words in visualization? Highlight components before speech.

\item Principles for assembly diagram: clear narrative sequence, components
    should be visible and identifiable, and spatial layout should be consistent
    from one frame to the next.

\item Use consistent representation from one part of a sequence to another.
    Similar views of 3D objects.

\item Use graphic devices to maintaint continuity from one view to another

\item Break down animated instructions into short meaningful segments.
    Allow users to play each separately.

\item Animation of human figures to show how to make specific movements

\end{compactenum}




\subsection{Chapter 10}

\e{Interacting with visualizations}

\begin{compactenum}

\item Fastest epistemic (knowledge or its validation) action: use hover
    queries. Only when query targets are dense and inadvertant queries aren't
    distracting.

\item Non-dominant hand should control frame-of-reference, dominant should
    make detailed selections

\item Moving objects on screen? Make sure the object moves in the same general
    direction as the hand

\item 3D navigation: sufficient number of objects must be visible to judge the
    relative view position.

\item Overview map to speed up acquisition of a mental map

\item Overview map to speed up large-scale navigation

\item Make each landmark visually distinct

\item Make landmarks recognizable as far as possible on all navigable scales

\item 3D map data: default controls should allow for tilt around horizontal
    and rotation around vertical, but not rotation around line of sight.

\item Overview map? Provide a ``you are here'' indicator

\item Navigation map should provide three views: north-up, track-up, track-up
    perspective (default)

\item Geometric fisheye distortion? allow maximum scale change of five

\item Geometric fisheye distortion? Meaningful patterns must always be
    recognizable.

\item Zooming interface? Default scaling rate: 3 to 4 per second. Allow this to
    be modified

\item  Large 2D/3D data spaces? Consider providing one or more windows to show a
    magnified part of space. In overview provide markers for the magnified
    views.

\end{compactenum}




\subsection{Chapter 11}

\e{Visual thinking process}

\begin{compactenum}

\item Design cognitive systems to maximize cognitive productivity

\item Interactive node-link diagram? Provide pathfinding algorithm for
    complex paths

\item In large data spaces with small islands of critical information,
    enable different viewports to show magnified areas. Useful for tasks that
    require more than 3 visual working memory chunks.

\end{compactenum}




%\end{multicols}




